The complexity class \CLS was introduced by Daskalakis and Papadimitriou in \cite{daskalakis2011continuous} with the goal of capturing the complexity of some well-known problems in $\PPAD \cap \PLS$ that have resisted, in some cases for decades, attempts to put them in polynomial time.  No complete problem was known for~\CLS, and in \cite{daskalakis2011continuous}, the problems \CM, i.e., the problem of finding an approximate fixpoint of a contraction map, and \PLCP, i.e., the problem of solving a P-matrix Linear Complementarity Problem, were identified as prime candidates. 

First, we present the first complete problem for \CLS, \MMCM, which is closely related to the problem \CM.

Second, we introduce \EOPL, which captures aspects of \PPAD and \PLS directly via a monotonic directed path, and show that \EOPL is in \CLS via a two-way reduction to \EOML. The latter was defined in~\cite{hubavcek2017hardness} to keep track of how far a vertex is on the \PPAD path via a restricted potential function, and was shown to be in \CLS.

Third, we reduce \PLCP to \EOPL, thus making \EOPL and \EOML at least as likely to be hard for \CLS as \PLCP. This result leverages the monotonic structure of Lemke paths for \PLCP problems, making \EOPL a likely candidate to capture the exact complexity of \PLCP; we note that the structure of Lemke-Howson paths for finding a Nash equilibrium in a two-player game directly motivated the definition of the complexity class \PPAD, which ended up capturing this problem's complexity exactly.

Finally, we reduce the 2-dimensional version of \CM to \EOPL, providing further evidence that \EOPL is \CLS-hard.
