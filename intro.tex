\chapter{Introduction}

The complexity class \TFNP, which stands for total function problems in
\NP, contains search problems that are guaranteed to have a solution, and whose
solutions can be verified in polynomial time~\cite{megiddo1991total}.
%
While \TFNP is a semantically defined complexity class and is thus unlikely to
contain complete problems, a number of syntactically defined subclasses of
\TFNP have proven very successful at capturing the complexity of total search 
problems.
% PPAD
For example, the complexity class $\cc{PPAD}$, introduced in
\cite{papadimitriou1994complexity} to capture the difficulty of search problems
that are guaranteed total by a parity argument, attracted intense attention in
the past decade culminating in a series of papers showing that the problem of
computing a Nash-equilibrium in two-player games is $\cc{PPAD}$-complete
\cite{chen2009settling,daskalakis2009complexity}. There are no known
polynomial-time algorithms for $\cc{PPAD}$-complete problems, and recent work
suggests that no such algorithms are likely to exist~\cite{bitansky2015cryptographic,garg2016revisiting}. 
% PLS
The class of problems that
can be solved by local search (in perhaps exponentially-many steps), $\cc{PLS}$,
has also attracted much interest since it was introduced in
\cite{johnson1988easy}. It looks similarly unlikely to have polynomial-time
algorithms. Examples of problems that are complete for \PLS include the problem
of computing a pure Nash equilibrium in a congestion
game~\cite{fabrikant2004complexity} and computing a locally optimal max
cut~\cite{schaffer1991simple}.

% Now CLS
If a problem lies in both \PPAD and \PLS then it is unlikely to be complete for 
either class, since this would imply an extremely surprising containment of one class in the other.
Motivated by the existence of several total function problems in \PPADPLS
that have resisted researchers' attempts to design polynomial-time algorithms,
in their 2011 paper \cite{daskalakis2011continuous} Daskalakis and Papadimitriou introduced
the class $\CLS$, a syntactically defined subclass of $\cc{PPAD} \cap \cc{PLS}$.
\CLS is intended to capture the class of optimization problems over a continuous
domain in which a continuous potential function is being minimized and the
optimization algorithm has access to a continuous improvement function.
Daskalakis and Papadimitriou showed that many classical problems of unknown
complexity were shown to be in $\CLS$ including the problem of solving a simple
stochastic game, the more general problem of solving a Linear Complementarity
Problem with a P-matrix, and the problem of finding an approximate fixpoint
to a contraction map. Moreover, $\CLS$ is the smallest known subclass of
$\cc{TFNP}$ without polynomial-time algorithms and hardness results for it imply hardness results for
$\cc{PPAD}$ and $\cc{PLS}$ simultaneously. 

%However, the original paper 
%of Daskalkakis and Papadimitriou did not prove that any natural problem was
%\CLS-complete.
%prove that any of the problems that 
%they showed to be in \CLS were hard for it.

Recent work by Hub\'{a}\v{c}ek and Yogev~\cite{hubavcek2017hardness} proved cryptographic 
lower bounds for \CLS. They introduced
a problem known as $\EOML$ which they showed was in $\CLS$, and for which they
proved a query complexity lower bound of $\Omega(2^{n/2}/\sqrt{n})$ and
hardness under the assumption that there were one-way permutations and
indinstinguishability obfuscators for problems in $\cc{P_{/poly}}$.
%
Another recent result showed that the search version of the Colorful
Carath\'eodory Theorem is in $\cc{PPAD} \cap \cc{PLS}$, and left open
whether the problem is also in $\CLS$~\cite{colorfulcara2017}.

Unfortunately \CLS is not particularly well-understood, and a glaring
deficiency is the current lack of any complete problem for the class. In their
original paper, Daskalakis and Papadimitriou suggested two natural candidates
for complete problems for \CLS, namely \CM and \PLCP, and this remains an
open problem.
Another motivation for studying these two problems is that 
solving Condon's simple stochastic games can be reduced to each of them
(separately) in polynomial time
and, in turn, there is sequence of polynomial-time reductions from parity games 
to mean-payoff games to discounted games to simple stochastic 
games~\cite{puri1996theory,gartner2005simple,jurdzinski2008simple,zwick1996complexity,hansen2013complexity}.
The complexity of solving these problems is unresolved and has received 
much attention over many years~(see, for example, 
\cite{zwick1996complexity,condon1992complexity,fearnley2010linear,jurdzinski1998deciding,bjorklund2004combinatorial,fearnley2016complexity}).
In a recent breakthrough, a quasi-polynomial time algorithm for parity games
was presented~\cite{parity}.
For mean-payoff, discounted, and simple stochastic games, the best-known 
algorithms run in subexponential time~\cite{ludwig1995subexponential}.
The existence of a polynomial time algorithm for solving any of these games
would be a major breakthrough.
For \ContractionMap and \PLCP no subexponential time algorithms 
are known, and providing such algorithms would be a major breakthrough.
As the most general of these problems, and thus most likely to be 
\CLS-hard, we study \ContractionMap and \PLCP.

Before presenting the results, we introduce the necessary definitions which will be used throughout the thesis. 

\section{Preliminaries}

In this section, we define polynomial-time reductions between total search problems
and the complexity classes \PPAD, \PLS, and \CLS.
%
\begin{definition}
For total functions problems, a (polynomial-time) reduction from problem~$A$ to
problem $B$ is a pair of polynomial-time functions $(f,g)$, such that $f$ 
maps an instance $x$ of $A$ to an instance $f(x)$ of $B$, and $g$ maps
any solution $y$ of $f(x)$ to a solution $g(y)$ of $x$.
\end{definition}
%
The complexity class $\PPAD \subset \TFNP$ is the class of problems reducible to the following problem:
\begin{definition}[\EOL~\cite{papadimitriou1994complexity}]
  Given two boolean circuits $\Succ, \Pred: \Set{0,1}^n\to \Set{0,1}^n$ such that $\Pred(0^n) = 0^n$, and $\Succ(0^n) \neq 0^n$ find some $x\in \Set{0,1}^n$ such that $x \neq 0^n$ and 
  \begin{enumerate}[leftmargin=*,label=(P\arabic*)]
    \item $\Pred(\Succ(x)) \neq x$, or
    \item $\Succ(\Pred(x)) \neq x$.
  \end{enumerate}
\end{definition}

The complexity class $\PLS \subset \TFNP$ is the class of problems reducible to the following problem:
\begin{definition}[\textsc{FindSink}~\cite{johnson1988easy}]
  Given a boolean circuit $\Neighbors$ mapping a vertex $\Set{0,1}^n$ to a set of at most $d$ neighbors for some constant $d$, and an arithmetic circuit $\Value : \Set{0,1}^n \to \R$, find a vertex $x\in \Set{0,1}^n$ such that for all $y\in \Neighbors(x)$, $\Value(x) \geq \Value(y)$.
\end{definition}


Following~\cite{daskalakis2011continuous}, we define the complexity class $\CLS$
as the class of problems that are reducible to the following problem \CLO.

\begin{definition}[\CLO~\cite{daskalakis2011continuous}]
\label{def:CLO}
Given two arithmetic circuits computing functions $f : [0,1]^3\to [0,1]^3$ and $p :
[0,1]^3 \to [0,1]$ and parameters $\e, \lambda > 0$, find either:
\begin{enumerate}[leftmargin=*,label=(C\arabic*)]
\item a point $x\in [0,1]^3$ such that $p(x) \leq p(f(x)) - \e$ or \label{c_fixpoint}
\item a pair of points $x,y\in [0,1]^3$ satisfying either \label{c_violation}
  \begin{enumerate}[label=(C\arabic{enumi}\alph*)] 
  \item $\Norm{f(x) - f(y)} > \lambda \Norm{x-y}$ or \label{c_bad_f}
  \item $\Norm{p(x) - p(y)} > \lambda \Norm{x-y}$. \label{c_bad_p}
  \end{enumerate}
\end{enumerate}
\end{definition}

In Definition~\ref{def:CLO}, $p$ should be thought of as a \emph{potential}
function, and $f$ as a \emph{neighbourhood} function that gives a candidate
solution with better potential if one exists. Both of these functions are 
purported to be Lipschitz continuous. A solution to the problem is either an approximate
potential minimizer or a witness for a violation of Lipschitz continuity.

\begin{definition}[\CM~\cite{daskalakis2011continuous}] \label{def:contractionmap}
We are given as input an arithmetic circuit computing $f: [0,1]^3\to [0,1]^3$,
a choice of norm $\Norm{\cdot}$, constants \mbox{$\e,c \in (0,1)$},
and $\delta > 0$, and we are promised that $f$ is $c$-contracting w.r.t. $\Norm{\cdot}$.
The goal is to find
\begin{enumerate}[label=(CM\arabic*)]
\item a point $x\in [0,1]^3$ such that $d(f(x),x) \leq \delta$, 
\item or two points $x,y\in [0,1]^3$ such that $\Norm{f(x) - f(y)}/\Norm{x-y} > c$. 
\end{enumerate}
\end{definition}

In other words, the problem asks either for an approximate fixed point of $f$ or
a violation of contraction. As shown in~\cite{daskalakis2011continuous}, \CM is
easily seen to be in \CLS by creating instances of \CLO with $p(x) =
\Norm{f(x)-x}$, $f$ remains as $f$, Lipschitz constant $\lambda = c+1$, and $\epsilon =
(1-c)\delta$.

\section{Our Contribution}

The main contributions of this thesis are: the first \CLS-complete problem, a new problem \EOPL in \CLS which we show to be equivalent to a hard problem in \CLS, and reductions from variants of both the problems offered as potential \CLS-complete problems to \EOPL.

In the problem $\ContractionMap$ (Definition \ref{def:contractionmap}), we
are asked to find an approximate fixed point of a function $f$ that is purported
to be contracting with respect to a metric induced by a norm (where the choice of
norm does not matter but is not part of the input), or to give a violation of
the contraction property for $f$. We introduce a problem, \MMCM, that allows
the specification of a purpoted meta-metric $d$ as part of the input of the problem,
along with the function $f$. We are asked to either find an approximate fixed
point of~$f$, a violation of the contraction property for~$f$ with respect to
$d$, a violation of the Lipschitz continuity of $f$ or $d$, or a witness that $d$ violates 
the meta-metric properties.
We show that \MMCM is \CLS-complete, thus identifying a first natural \CLS-complete problem.
We note that, contemporaneously and independently of our work, Daskalakis, Tzamos, and
Zampetakis~\cite{DTZ17} have defined the problem \MBanach and shown it is \CLS-complete.
Their \CLS-hardness reduction produces a metric and 
is thus stronger than our \CLS-hardness result for \MMCM. 
We discuss both results in more detail in Section~\ref{sec:MMCMisCLScomplete}.

Our second result is to show that a new problem \EOPL is equivalent to \EOML.
The \EOML problem was introduced to capture problems that have a PPAD directed
path structure while that also allow us to keep count of exactly how far the vertex is from
the start of the path. In a sense, this may seem rather unnatural, as many common
problems do not seem to have this property. In particular, while the \PLCP problem
has a natural measure of progress towards a solution given by Lemke's algorithm, 
this is given in the form of a potential function, rather than an exact measure
of the number of steps from the beginning of the algorithm.

To address this concern, we introduce a new problem \EOPL which captures problems with a
PPAD path structure that also have a potential function that decreases
along this path. It is straightforward to show that \EOPL is more general than
\EOML. However, despite its generality, we are able to show that \EOPL can
be reduced to \EOML in polynomial time, and so the two problems are equivalent
under polynomial time reductions.


We then show that \PLCP can be reduced to \EOPL,
which provides an alternative proof that $\PLCP$ is in $\CLS$.

Our final result is to show that the 2-dimensional variant of \CM (\TwoDContractionMap) can be reduced to \EOPL. In light of the simultaneous results, here and in \cite{DTZ17}, showing that multiple variants of \CM are \CLS-complete, this is a particularly suggestive result, and opens up a what we believe will be a fruitful line of inquiry.

We believe that the \EOPL problem is of independent interest, as it naturally
unifies the circuit-based view of $\cc{PPAD}$ and of $\cc{PLS}$, and is defined
in the spirit of the canonical definitions of $\cc{PPAD}$ and $\cc{PLS}$.  There
are two obvious lines for further research.

Given the reductions we provide,
\EOPL is a more likely candidate for \CLS-completeness than either $\PLCP$ or $\CM$, and it seems plausible that the reduction from \TwoDContractionMap to \EOPL can be generalized to yield a proof of \CLS-completeness for \EOPL.

Alternatively, one could attempt to reduce \EOPL to \PLCP, thereby showing that
that \PLCP is complete for the complexity class defined by these two problems,
and in doing so finally resolve the long-standing open problem of the complexity
of \PLCP. We note that, in the case of finding a Nash equilibrium of a two-player game,
which we now know is
\PPAD-complete~\cite{chen2009settling,daskalakis2009complexity}, the definition
of \PPAD was inspired by the path structure of the Lemke-Howson algorithm, as
our definition of \EOPL is directly inspired by the path structure of Lemke
paths for P-matrix LCPs.

\section{Technical Overview}
In Chapter \ref{sec:MMCMisCLScomplete} we show that \MMCM is \CLS-complete. In Chapter \ref{sec:EOMLtoEOPL} we introduce \EOPL, and show that it is equivalent to \EOML. In Chapter \ref{sec:PLCPtoEOPL} we reduce \PLCP to \EOPL. Finally, in Chapter \ref{sec:twoDContractionMapToEOPL} we reduce the two-dimensional variant of \CM to \EOPL.

