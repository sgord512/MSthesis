\chapter{\EOML to \EOPL and Back}
\label{sec:EOMLtoEOPL}

In this section, we define a new problem \EOPL.
Then, we design polynomial-time reductions from \EOML to \EOPL, and
from \EOPL to \EOML, thereby showing that the two problems 
are polynomial-time equivalent. In Section~\ref{sec:PLCPtoEOPL},
we reduce \PLCP to \EOPL.

First we recall the definition of \EOML, which was
first defined in~\cite{hubavcek2017hardness}.
It is close in spirit to the problem \EOL that is used
to define \PPAD~\cite{papadimitriou1994complexity}. 

\begin{definition}[\EOML~\cite{hubavcek2017hardness}]
Given circuits $S,P: \{0,1\}^n \rightarrow \{0,1\}^n$, and $V:\{0,1\}^n\rightarrow \{0,\dots, 2^n\}$ such that $P(0^n) =0^n\neq S(0^n)$ and $V(0^n)=1$, find a string $\xx \in \{0,1\}^n$ satisfying one of the following 
\begin{enumerate}[label=(T\arabic*)]
\item either $S(P(\xx))\neq \xx \neq 0^n$ or $P(S(\xx))\neq \xx$,
\item $\xx\neq 0^n, V(\xx)=1$,
\item either $V(\xx)>0$ and $V(S(\xx))-V(\xx)\neq 1$, or $V(\xx)>1$ and $V(\xx)-V(P(\xx))\neq 1$. 
\end{enumerate}
\end{definition}
Intuitively, an \EOML is an \EOL instance that is also equipped with an
``odometer'' function. The circuits $P$ and $S$ implicitly define an
exponentially large graph in which each vertex has degree at most 2, just as in \EOL, and condition T1 says that the end of
every line (other than $0^n$) is a solution.
In particular, the string 
$0^n$ is guaranteed to be the end of a line, and so a solution can be found by
following the line that starts at $0^n$.
 The function $V$ is intended to help with this, by giving the number of steps
that a given string is from the start of the line. We have that $V(0^n) = 1$,
and that $V$ increases by exactly 1 for each step we make along the line.
Conditions T2 and T3 enforce this by saying that any violation of the property
is also a solution to the problem. 


In \EOML, the requirement of incrementing $V$ by exactly one as walk along the
line is quite restrictive. We define a new problem, \EOPL,  which is similar in
spirit to \EOL, but drops the requirement of always incrementing the potential
by one as we move along the line.

\begin{definition}[\EOPL]
\label{def:EOPL}
Given Boolean circuits $S,P : \Set{0,1}^n \to \Set{0,1}^n$ such that $P(0^n) =0^n\neq S(0^n)$ and a Boolean circuit $V: \Set{0,1}^n \to \Set{0,1,\dotsc,2^m - 1}$ such that $V(0^n) = 0$ find one of the following:
\begin{enumerate}[label=(R\arabic*)]
\item A point $x \in \Set{0,1}^n$ such that $S(P(x)) \neq x \neq 0^n$ or $P(S(x)) \neq x$.
\item A point $x \in \Set{0,1}^n$ such that $x \neq S(x)$, $P(S(x)) = x$, and $V(S(x)) - V(x) \leq 0$.
\end{enumerate}
\end{definition}

The key difference here is that the function $V$ is required to be strictly
monotonically increasing as we walk along the line, but the amount that it
increases in each step is not specified.
At first glance, the definition of \EOPL may seem more general and more likely to 
capture the whole class \CLS. In fact, we will show that \EOML and \EOPL are 
inter-reducible in polynomial-time.
%
\begin{theorem}
\EOML and \EOPL are equivalent under polynomial-time reductions.
\end{theorem}
%
As expected, the reduction from~\EOML to \EOPL is relatively easy. It requires
handling the difference in potential at $0^n$ and vertices with potential zero that
are not discarded directly as possible solutions in \EOPL. We make the latter
self loops, but that creates extra starts and ends of lines which need to be
handled. Full details of the reduction with proofs are in
Appendix~\ref{sec:EOMLtoEOPL}.

The reduction from \EOPL to \EOML is involved, and appears in detail in
Appendix~\ref{sec:eopl2eoml}. Here the basic idea is to insert missing single
increments in between by introducing new vertices along the original edges. To
allow this we need to encode potential itself in the vertex description. If
there is an edge from $\uu$ to $\uu'$ in the \EOPL instance whose respective
potentials are $p$ and $p'$ such that say $p<p'$ then we create edges $(u,p)\ra
(u,p+1)\ra \dots \ra (u,p'-1)\ra (u,p')$. However, this creates a lot of dummy
vertices, namely those that never appear on any edge due to irrelevant potential
values, i.e., in this example $(u,\pi)$ with $\pi <p$ or $\pi\ge p'$. We make
them self loops (not an end-of-line) with zero potential, and since
non-end-of-line solutions of \EOML, namely $T2$ and $T3$, must have strictly
positive potential, these will never create a solution of the \EOML instance.

In addition, a number of issues need to be handled with consistency: $(a)$
a $T2$ type solution of \EOML may be neither at the end of any line nor be a 
potential violation in \EOPL; we do extra (linear time) work to handle such
solutions, $(b)$ a $T3$ type potential violation may not be on a ``valid'' edge as
required by \EOPL. $(c)$ ``invalid'' edges, $(d)$ potential difference at the
initial vertex $0^n$, etc.

% we make these dummy solutions self loops (non-end-of-line) with zero potential.} 

%which we handle by creating self loops and assigning zero potential. The fact that non-end-of-line solutions of \EOML, namely $T1$ and $T2$, must have strictly postive potential ensures no extra . }

%The easier and expected reduction is from~\EOML to \EOPL. There are some details 
%around the start and ends of the line. For \EOPL to \EOML, the basic idea is to
%insert missing single increments in between. Both reductions appear with all details 
%in Appendices~\ref{sec:EOMLtoEOPL} and~\ref{sec:eopl2eoml}, respectively.
